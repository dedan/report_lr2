\documentclass{beamer}

\usepackage[utf8]{inputenc}
\usepackage[T1]{fontenc}
\usepackage{listings}

%% Choose a theme from one of the following:
% AnnArbor, Antibes, Bergen, Berkeley, Berlin, Boadilla, boxes, CambridgeUS
% Copenhagen, Darmstadt, default, Dresden, Frankfurt, Goettingen, Hannover
% Ilmenau, JuanLesPins, Luebeck, Madrid, Malmoe, Marburg, Montpellier,
% PaloAlto, Pittsburgh, Rochester, Singapore, Szeged, Warsaw
\usetheme{Berlin}

\setbeamercovered{transparent}

\hypersetup{pdfpagelabels=true}

\lstset{ 
language=Haskell,
breaklines=true,   
basicstyle=\ttfamily,
keywordstyle=\color{keywordcolor},
        commentstyle={\color{commentcolor}\itshape},
        stringstyle={\color{stringcolor}\underbar},
        identifierstyle=\color{idcolor},numbers=left,
        xleftmargin=2em,framerule=0.8pt,
        stepnumber=1,frame=tlrb,showstringspaces=false,
        firstnumber=1,numberstyle=\ttfamily,backgroundcolor=\color{bg}}


%% Choose one of the following color themes or go with the default
% albatross, beaver, beetle, crane, default, dolphin, dove, fly
% lily, orchid, rose, seagull, seahorse sidebartab, structure
% whale, wolverine
\usecolortheme{beaver}

\title[Proto-Objects]{Proto-object based attention}
\subtitle{and the application to computer vision}

\author[Stephan Gabler] { \\\texttt{stephan.gabler@gmail.com}} 
\date[06/2011] {\today}

%\pgfdeclareimage[height=0.6cm]{university-logo}{Logo_Links_rot}
%\logo{\pgfuseimage{university-logo}}

\beamerdefaultoverlayspecification{<1->}



\begin{document}

\frame{\titlepage}

\begin{frame}
    \frametitle{Attention}
    \framesubtitle{What is attention?}
    \begin{quotation}
        Everyone knows what attention is. It is the taking possession of the mind, in clear and vivid form, of one out of what seem several simultaneous possible objects or trains of thought. Focalization, concentration of consciousness are of its essence. It implies withdrawal from some things in order to deal effectively with others, and is a condition which has a real opposite in the confused, dazed, scatterbrain state… – William James (1890)
    \end{quotation}
\end{frame}


\begin{frame}
    \frametitle{Attention}
    \framesubtitle{Why attention?}
    
    \begin{itemize}
        \item<1-> We don't have the capacity to process all stimuli to the same extent
        \item<2-> Historically (Broadbent's filter theory - 1958) it is thought that attention serves as early selection.
    \end{itemize}
\end{frame}


\begin{frame}
    \frametitle{Attention}
    \framesubtitle{Different kinds of attention}
    \begin{itemize}
        \item low-level -> bottom up ==> stimulus dependend
    \end{itemize}
    
\end{frame}


\begin{frame}
    \frametitle{Attention}
    \framesubtitle{High-level attention -> proto-objects}
    
\end{frame}


\begin{frame}
    \frametitle{Idea}
    \framesubtitle{Use high-level attention in computer vision}
    
\end{frame}









\end{document}
