%!TEX root = /Users/dedan/bccn/lab_rotations/johannes/report/report.tex
\begin{abstract}

\section*{Abstract}

In every second the brain has to deal with thousands of stimuli that have to be processed, stored and reacted on. But not all of them reach our consciousness and it is thought that not even all of them are processed to the same extent. The mechanism which is responsible for this selection is called attention and works as a selection mechanism or preference rule for incoming stimuli. For a long time the majority opinion was that attention is directed by low-level features of the stimulus, e.g. luminance contrast in the case of visual stimuli. The idea was that in a first stage many low-level features are processed in parallel before more complex tasks as for example object recognition are done in a serial stage. But recent research \cite{Einhauser:2008cv} showed that attention might as well be object based and this rises the question: \emph{How can we attend to objects before they are recognized as objects?}. One answer to this question might be that mid-level features like continuation, boundaries and occlusion might serve to compute so-called proto-objects - areas of potential objects. The objective of this work is whether this proto-objects can also be used in computer vision and if it is possible to come up with a simple heuristics for detecting them. To achieve this the depth information of a 3D Camera was used in combination with a simple histogram clustering approach which allowed to look for \emph{interesting layers} of the depth image. This method makes it possible to restrict the computationally expensive high-level object detection (SIFT-features) to much smaller patches which increased the speed of computation. The method also decreased the amount of false-positive matches because  the shape of the proto-object could be used as a mask on the image patch on which the features where computed. This avoided the common problem that often features which actually belong to something in the background are mistakingly attributed to the object. This work shows that it is possible to successfully apply the concept of proto-objects in computer vision.

\end{abstract}
